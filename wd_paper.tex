% This is LLNCS.DEM the demonstration file of
% the LaTeX macro package from Springer-Verlag
% for Lecture Notes in Computer Science,
% version 2.4 for LaTeX2e as of 16. April 2010
%
\documentclass{llncs}
%
\usepackage{makeidx}  % allows for indexgeneration
%
\begin{document}
%
\frontmatter          % for the preliminaries
%
\pagestyle{headings}  % switches on printing of running heads
\addtocmark{Hamiltonian Mechanics} % additional mark in the TOC
%
%
\mainmatter              % start of the contributions
%
\title{Evaluation of External References in a Collaborative Knowledge Base}
%
\titlerunning{Evaluation of External References}  % abbreviated title (for running head)
%                                     also used for the TOC unless
%                                     \toctitle is used
%
\author{Alessandro Piscopo\inst{1} \and Roger Temam\inst{2}
Jeffrey Dean \and David Grove \and Craig Chambers \and Kim~B.~Bruce \and
Elsa Bertino}
%
\authorrunning{Ivar Ekeland et al.} % abbreviated author list (for running head)
%
%%%% list of authors for the TOC (use if author list has to be modified)
\tocauthor{Ivar Ekeland, Roger Temam, Jeffrey Dean, David Grove,
Craig Chambers, Kim B. Bruce, and Elisa Bertino}
%
\institute{University of Southampton, Southampton Hampshire, UK,\\
\email{A.Piscopo@soton.ac.uk},\\
\and
Universit\'{e} de Paris-Sud,
Laboratoire d'Analyse Num\'{e}rique, B\^{a}timent 425,\\
F-91405 Orsay Cedex, France}

\maketitle              % typeset the title of the contribution

\begin{abstract}
The abstract should summarize the contents of the paper
using at least 70 and at most 150 words. It will be set in 9-point
font size and be inset 1.0 cm from the right and left margins.
There will be two blank lines before and after the Abstract. \dots
\keywords{computational geometry, graph theory, Hamilton cycles}
\end{abstract}
%
\section{Introduction}
%
Where we discuss why structured KBs are important and why it is worth to study external references.\\
Wikidata, what is it?\\
References in Wikidata\\
Vision\\
Mission\\
Steps\\
3 contributions\\

%
\section{Related work}
%
http://www.korfiatis.info/webpagen/wp-content/papercite-data/pdf/2006-OIR.pdf

\section{Framework for discovery and evaluation}
Methods\\
Discovery\\
    Quality
    
\section{Experimental evaluation}
    Model\\
    Experiments
    
\section{Conclusion}
Some conclusions

%
% ---- Bibliography ----
%
\nocite{*}
\bibliographystyle{splncs03}
\bibliography{wd_sources}


\end{document}
